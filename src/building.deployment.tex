\chapter{Building and Deployment}
\label{appendix:building.deployment}

This appendix discusses the steps necessary to get Warlock running.%
\sidenote[7]{
  Warlock was not open source at the time of writing the thesis, though there
  were plans around it. The interface and commands will most probably remain
  the same, allowing us to use this section.
}

\section{Building}

Download, build and start Warlock.

\begin{scode}{bash}{a.build.code}{%
  Build Warlock}{%
  Download and build Warlock.}
  \begin{lstlisting}
# Get the source  
$ git clone https://github.com/wooga/warlock

# Build the code
$ cd warlock
$ make
$ make rel

# Create release
$ cd rel/warlock
$ bin/warlock start
  \end{lstlisting}
\end{scode}

\section{Development Cluster}

Build a simple 3 node development cluster.

\begin{scode}{bash}{a.dev.code}{%
  Create Warlock dev cluster.}{%
  Create Warlock dev cluster.}
  \begin{lstlisting}
# Create dev release
$ make devrel

# Start all the nodes
$ dev/dev1/bin/warlock start
$ dev/dev2/bin/warlock start
$ dev/dev3/bin/warlock start

# Connect any of the two node with the other node, creating the cluster
$ dev/dev2/bin/warlock-admin join warlock1@127.0.0.1
$ dev/dev3/bin/warlock-admin join warlock1@127.0.0.1
  \end{lstlisting}
\end{scode}

\section{Command execution}

Execute commands on Warlock with ETS as back end.

\begin{scode}{bash}{a.comm.exex}{%
  Executing commands on Warlock.}{%
  Executing commands on Warlock.}
  \begin{lstlisting}
# Connect to dev node 1 and run a set command  
$ dev/dev1/bin/warlock attach

(warlock1@127.0.0.1)1> war_server:x(clu, [set, key, value]).
{ok,success}

# Connect to dev node 2 and try to get the value set before
$ dev/dev2/bin/warlock attach

(warlock2@127.0.0.1)1> war_server:x(loc, [get, key]).
{ok,value}
  \end{lstlisting}
\end{scode}
