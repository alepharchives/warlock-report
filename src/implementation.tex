\chapter{Implementation}
\label{chapter:implementation}

We detail the implementation of the system as per the requirements
(\chapterref{requirements}) and
design (\chapterref{analysis.design}) in this chapter. We also explain how the
design goals matches on to the features of Erlang programming
language and \abbr{OTP} libraries.

The choice of Erlang as the programming language for implementation was
motivated by the factor that the Magic Land back end was already implemented in
Erlang. The idea is that a new system in the same language will be much easier
to integrate with the existing architecture.

The development of the system was done in iterations%
\sidenote{
  \emph{Iteration}: refers to the repeated execution of a development process.
  A typical process involves requirement analysis, design, implementation and
  testing. This is along the lines of Agile development \citep{Kinoshita}.
}, roughly a week in size.
The target of each iteration was to extend the existing base or to develop
new features. This allowed us to quickly identify the part of the system that
forms the core and other features that are just good to have.

\section{Prototype}

The core of the consensus component is a variation of
the Paxos protocol formulated in \citet{Robbert2011}. While
the algorithm is straight forward, implementation
oriented and covers more edge cases than \citet{Lamport01}, a working prototype
helps to get a more through understanding of the protocol's intricacies.

We built a prototype that is a straight one-to-one implementation of the
pseudo code from \citet{Robbert2011}. It supports a configurable number of
replicas, leaders and acceptors, allowing us to get an idea of how the protocol
behaves with different configurations using simple unit tests. Building the
prototype helps us single out the core of the system and make sure the
protocol works correctly before getting to the other parts.

\section{Algorithm Implementation}

Each of the processes described in \sectionref{concepts.paxos} has some state
and has to send and receive messages to and from other processes. We use
\term{gen\_server}s
\sectionref{gen.server} to implement these processes since process roles map
on to this behaviour.

\term{gen\_server}s can internally maintain state that can be accessed using
all of its
callback functions. It also provides functions to send messages%
\sidenote{
  \term{gen\_server} messages can either be call (synchronous) or cast
  (asynchronous).
}
to one or more processes. The callback functions \term{handle\_cast}%
\sidenote[3]{
  \emph{handle\_cast}: A \term{gen\_server} callback function
  used to handle asynchronous messages. It cannot reply to the caller
  directly, but must use \term{\term{gen\_server}:reply/2} function if it
  wants to do so.
}, \term{handle\_call}%
\sidenote[8]{
  \emph{handle\_call}: A \term{gen\_server} callback function used to handle
  synchronous messages. It has the caller's address and blocks the caller
  until it sends a reply.
} and \term{handle\_info}%
\sidenote[13]{
  \emph{handle\_info} is the \term{gen\_server} callback function used to
  handle all the other types of messages, e.g. plain Erlang messages,
  \abbr{TCP}, \abbr{UDP} messages and so on.
}
are used to handle incoming messages.

Warlock uses \term{handle\_cast} instead of \term{handle\_call} since the
protocol is
for a system that communicates with asynchronous messaging. This also allows us
to decouple the processes and take a step towards creating a lock-free system.

Let us take a look at the individual \term{gen\_server} processes. Note
that all of these processes reside on the same node inside an
Erlang application. For each of the process roles, we describe the state
it maintains and its the main responsibilities.

\subsection{Replica}

The replicas are responsible for maintaining the state of the system. They
represent the state machine that updates based on the events sent to it. We
have state machine replication by making sure we execute the events in the same
sequence across all the nodes in the cluster.

In the original algorithm, the replica maintains a transaction log. The goal of
the protocol is to keep this log consistent across multiple nodes. Instead of
maintaining such a log, we execute the agreed commands (using callback functions).

\subsubsection{State}

\begin{itemize}
    \iterm{Current slot number}: The strictly increasing property of slots
    can be used to ensure that the set of decisions taken by
    the algorithm are sequential and ordered.

    The current slot number is the next open slot available for a decision.
    \iterm{Minimum slot number}: The protocol allows for concurrent proposals.
    Each of the incoming proposals are allotted the next available slot and
    kept there until they are decided upon and committed.

    The minimum slot number is the next available slot for a new incoming
    proposal.
    \iterm{Proposals}: A table that maps proposals to slot numbers. It is
    implemented as a hash table.
    \iterm{Decisions}: It is sometimes possible that we receive decisions for
    higher numbered slots before the lower ones because the protocol is
    asynchronous and concurrent. Since it is also sequential, we have to wait
    until we commit all the lower numbered slots before we can proceed.

    Decisions is a table that maps decided proposals to slot numbers and is
    implemented as a hash table.
\end{itemize}

\subsubsection{Role}

\begin{itemize}
  \item Receives proposals the leader, assigns slot numbers to them and
    forwards them to the leader.
  \item Receives decisions from the leader and commits them sequentially.
\end{itemize}

\subsection{Acceptor}

The acceptors form the \term{memory} of the protocol and ballots is the
mechanism used for it.

\subsubsection{State}

\begin{itemize}
    \iterm{Ballot Number}: The maximum ballot number it has seen in all
    of the messages it has received.
    \iterm{Accepted}: The leader sends a proposal to the acceptor along with
    its ballot for a specific slot. The acceptors stores the proposal for
    the maximum ballot it has seen for that specific slot. It is implemented as
    a hash table.
\end{itemize}

\subsubsection{Role}

\begin{itemize}
  \item Responds to \code{P1A} messages with the \code{P1B} message,
    maximum ballot and the entire accepted table.
  \item Responds to \code{P2A} messages with the \code{P2B} message
    along with maximum ballot. If the ballot in the \code{P2A} message
    was equal to or great than the one in its state, it adds it as
    an entry in the accepted table.
\end{itemize}

\subsection{Scout}

The scout is a process spawned by the leader to handle the first phase of the
protocol.

\subsubsection{State}
\label{section:scout.state}

\begin{itemize}
    \iterm{Leader}: Address of the leader that spawned it.
    \iterm{PValues}:%
    \sidenote{
      A PValue is a tuple consisting of the ballot number, slot number and the
      proposal.
    }
    Unique set of PValues returned by the acceptor. It is implemented as a
    hash table.
\end{itemize}

\subsubsection{Role}

\begin{itemize}
  \item Initiates the first phase of the protocol by sending \code{P1A}
    message to all the acceptors in the group.
  \item Waits for replies from majority of acceptors while collecting
    all the PValues sent by them. It only maintains the PValues for slots
    with maximum ballots.
  \item Replies to the leader with the set of PValues if it gets majority
    acceptance.
\end{itemize}

\subsection{Commander}

The commander is a process spawned by the leader to handle the second part of
the protocol. It is very similar to the implementation of the scout. A commander
is spawned for every proposal.

\subsubsection{State}

Same as that of the scout, but instead of storing all the
PValues, it only stores the PValue corresponding to the proposal it is working
on.

\subsubsection{Role}

\begin{itemize}
  \item Initiates the second phase of the protocol by sending \code{P2A}
    messages to all the acceptors.
  \item It waits for replies from a majority of the acceptors.
  \item On receiving the required acceptance, it sends a message to all the
    replicas indicating that the proposal has been accepted.
\end{itemize}

\subsection{Leader}

The leader process co-ordinates votes for proposals using scouts and commanders.

\subsubsection{State}

\begin{itemize}
    \iterm{Ballot}: A ballot is unique to a leader and is used to tag all
    proposals being sent to the acceptors.
    \iterm{Active}: In multi-paxos, once the leader is in the second phase of
    the protocol, it does not need to run first phase of the protocol again.

    The \term{active} flag determines what phase of the protocol the leader is
    in.
    If it is set to true, the leader is in active phase allowing it to directly
    spawn commanders for every incoming proposal.
    \iterm{Proposals}: A mapping from slots to proposals. It is implemented as
    a hash table.
    \iterm{Timer Reference}: The leader uses a single timer based on its current
    status. This flag keeps track of it. Different possible values for the flag
    are
    \begin{itemize}
        \iterm{renew}: The leader needs to renew its lease before it expires if
        it is in the master node.
        \iterm{backoff}: When the leader's scout gets preempted, instead of
        immediately re-spawning the scout, the leader waits for a prespecified
        \term{backoff} time.
        \iterm{master\_check}: Periodic check by the leader to see if the
        existing master node's lease has expired. This is the check that is run
        before spawning a new scout.
        \iterm{membership}: A new node being added to the cluster has to wait
        until it gets accepted. This flag is used to poll its acceptance.
    \end{itemize}
    \iterm{Monitors}: Monitors are used by the master leader to keep track of
    the process status of leaders of the other nodes.
\end{itemize}

\subsubsection{Role}

\begin{itemize}
  \item Receives proposals from the replica and either queues or spawns a
    commander for it based on its active flag's value.
  \item Handles communication between scout and commander, spawning them
    when required and acting on the results sent by them.
  \item Handles all ballot related activities such as incrementing the ballot
    when preempted and incrementing the view with the change in the system
    configuration.
  \item Initiates and participates in master node elections.
  \item Receives and acts on the messages resulting from monitoring of other
    nodes.
  \item Uses a single timer based on its current status to poll for the
    reasons mentioned.
\end{itemize}

\subsection{Client}

The consensus client process is only used for reconfiguration requests.
The client behaves as a code library for the actual
proposals. The replica uses the client as a execution point for the decided
proposals.

\section{System Reconfiguration}
\label{section:impl.rcfg}

As discussed in \sectionref{a.n.d.reconfig}, we use a parallel state machine
to track the node's configuration. This state machine is implemented as a
hash table. The operations we can support with this design choice are:

\begin{itemize}
    \iterm{join}: A fresh node can join the consensus cluster without copying
    any data from the other nodes. This command is used to setup the initial
    cluster.
    \iterm{add\_repl\_member}: A new node can be added to the system by first
    copying data into it and then added to the cluster once it is in sync with
    rest of the member nodes.
    \iterm{leave}: A member node can leave the cluster.
    \iterm{remove}: Any of the members nodes can remove any of the other member
    node from the cluster.
    \iterm{transfer\_master}: Nodes are designed to retain their master status
    for as long as possible. This command can be used to transfer the master
    status to any other node.
\end{itemize}

The reconfiguration requests are treated as proposals allowing us to use the
same set of functions, albeit with different slots.

\section{Application Structure}

As described earlier, the code is split into different applications according
to their functionality. A tree view of the source structure is provided in
\appendixref{source.code} for reference.

\section{Building and Deployment}

A build%
\sidenote{
  \emph{Software Build}: Compiled set of source code ready to be run.
}
process is necessary for Warlock since it is made up of multiple Erlang
applications. We also need to create a small cluster of Warlock nodes during
development for observation and testing. This requires us to have an automated
procedure to generate and deploy builds.

The final build consists of a portable Erlang runtime system, log
paths setup and all the required libraries. rebar \citep{rebar}, reltool
\citep{reltool} and make files \citep{makefiles} are the tools used to enable
this automation.

\section{Testing}
\label{section:impl.testing}

The importance of testing increases when building a system that is developed
rapidly in small increments. Even though we start with a fixed set of
requirements, we might
need to changes different components of the system in order to improve
performance or add features. Having a solid set of tests allows
us to make bold changes to the code while minimizing the chances of breaking
working code.

\subsection{Unit Testing}

Testing individual units of code is unit
testing. While we did not write tests for each and every part of the code as
required by test driven development methodology \citep{Beck:2002:TDD:579193},
we wrote tests in a way that covered the critical code path.
Some parts of the code that act more as a library has unit tests for it.

\subsection{Development Cluster}

The system needs distributed system tests. We automate the creation of a test
cluster allowing us to quickly inspect, reproduce and trace bugs that appear
only when the system is running as a part of the cluster.

\section{Logging}

Logging of events in the system is essential since it is one of the primary
methods of debugging in a distributed system. It also allows us to track rare
failures and helps during system implementation.

We use lager \citep{lager}, a library for logging. It has the flexibility of
providing many log levels%
\sidenote{
  \emph{Log levels}: Log levels determines the severity of the event that is
  being logged allowing the developer to take appropriate action.
}
and multiple outputs (console logging, file based logging).

The logging systems are designed with performance in mind. Even logging can have
noticeable effects even if the log level is higher. The key is to balance the
points where events are logged and to keep them at a bare minimum when used in
live systems.

\section{Pluggable Back Ends}

The Warlock project was designed as a lock manager to replace Redis
\citep{redis}. While the default Warlock system supports the operations
required for the Magic Land back end, other projects using Warlock might need
some other commands. A better solution than re-implementing the commands is to
support multiple back ends.

So Warlock acts as a consensus layer and its callback can be used to run
commands on different back ends. We have implemented back ends in \abbr{ETS},
Redis and Judy arrays \citep{judy}. It is possible to use other software
as back end since Warlock is transparent to the commands.

\section{Performance}

While consistency is the primary goal, the system also needs to focus on performance
to handle systems with millions of users. Apart from the many optimizations
described in \chapterref{analysis.design}, we use some tools to locate
the bottlenecks and attempt to fix it without affecting the primary goal.

\subsection{Profiling}
\label{section:impl.profiling}

Profiling is analyzing the code dynamically based on different measurements such
as \abbr{CPU} time used, memory consumed, number of function calls made,
number of messages
sent and so on. It allows us to discover the parts of the source code that take
the longest to run or is run the most number of times.

Erlang has a number of built in specialized profilers available
\citep{erlang.profiling}. It helped us identify that at one point logging was
responsible for degrading upto 30\% of performance even though it was set to a
lower log level.

\subsection{Benchmarking}

Profiling allows us to look though the performance of the code minutely.
Benchmarking allows us to see how the system performs from an external
point of view. It is the act of testing the performance of a system by
repeatedly running it or interacting with it.

Basho bench \citep{basho.bench} is the benchmarking tool used for the project.
It allows us to define a number of options such as the number of concurrent
workers, functions to call, rate of calls, system status measurements and so
on. However, it was unable to generate enough traffic to saturate the system
on a network while running from a separate server. While distributed
benchmarking can be done to generate more calls, the coordination becomes harder
and it gets difficult to merge the results.

\subsection{Pluggable Hash Tables}

Hash tables are used across the implementation of Warlock making its
implementation choice a crucial factor from the performance perspective.
Erlang provides many options:

\begin{itemize}
    \iterm{ETS}: Erlang Term Storage is a built-in storage module in Erlang
    used mainly for storing tuples. It provides the option of multiple data
    structures and provides many atomic operations on them.
    \iterm{Dict}: Dict is Erlang's implementation of an immutable dictionary
    data structure.
    \iterm{gb\_trees}: Erlang's implementation of General Balanced Trees
    \citep{Andersson99}. It is an efficient tree structure that can be used
    as an hash table.
\end{itemize}

By using generalized hash tables in our code, we were able to evaluate these
data structures finally choosing \abbr{ETS} for its performance and features.

\section{Erlang Libraries}

A few external projects were used for building this system. We describe the
roles of these project.

\subsection{lager}

Lager \citep{lager} is a logging framework for Erlang applications. It provides
features such as fine grained log levels, multiple back end support, optimized
logging and relatively better performance.

\subsection{eredis}

eRedis \citep{eredis} is a non-blocking Erlang client library for Redis.

eRedis is used in this project to support the Redis back end and is used by the
project to provide Redis protocol support.

\subsection{Ranch}

Ranch \citep{ranch} is a socket acceptor pool for \abbr{TCP} protocols.

Ranch is used in this project for accepting and managing multiple client
\abbr{TCP} connections for the Redis protocol implementation.

