\chapter{Future Work}
\label{chapter:future.work}

The Warlock project in its current state is quite stable and ready to be used.%
\sidenote[6]{
  Based on load tests in chapter [[]]
}

\todo{Add details about Warlock's current usage}

\section{Security}

Warlock is meant to be run within the internal network and is currently lacking
in terms of security. While this makes it reasonably safe to run within a secure
network, improved security will allow it to run in more diverse environments.

The client based on Redis protocol supports basic authentication. However, it
needs improvements in terms of the password storage, configuration and
transmission.

\section{Consistency Checks}

We treat the database part of Warlock as the state machine and update it based
on the proposals that made through the protocol. However, it might be possible
that few of nodes become slow in executing these decisions, falling behind
rest of the cluster. If the decision queue grows too large, some of the
decisions may even be skipped. While, this has not been observed even under
heavy loads, it is a good idea to have a consistency hash based on the data
which is shared periodically with the cluster, providing us with an idea of
how many nodes are in sync and how many are catching up.

\section{Testing}

As discussed in \sectionref{impl.testing}, testing is essential for distributed
projects. While basic set of tests cover the common use cases and flows, more
tests help us to cover the corner cases better.

Warlock can benefit from PropEr%
\sidenote{
  PropEr \citep{proper} is a property based testing tool for Erlang.
}
by providing automated tests and can also help in improving the overall
system stability. By just specifying the structure of the program and the range
of input values, PropEr can generate the required number of tests needed for
us to gain confidence in our implementation. This tool can then be made
part of compile phase itself, allowing us to catch discrepancies at a much
earlier stage.

System tests%
\sidenote{
  \emph{System Testing}: Testing the system as a whole, usually treating it as 
  a black box (ignoring the implementation details).
}
become necessary in distributed systems to make sure changes made to the code
does not break any previous features. Common test \citep{common.test} is a
system testing tool that is a part of the Erlang distribution. It is feature
rich and allows us to implement system and integration tests. This tool can
be used to run system tests during the build phase, thus helping to make sure
the build works are expected on multiple nodes.

\section{Performance}

Performance is crucial for projects serving requests for millions of users.
While Warlock if tuned to provide necessary performance for Magic Land, it
can still be tweaked to push it even further.

\subsection{Code Profiling}

Profiling, as explained in \sectionref{impl.profiling}, can help us identify the
bottlenecks in the codebase. Erlang comes with lots of different profilers
which can be used to identify bottlenecks from different perspectives. This
should allow us to focus more on the most frequently used functions and try
to reduce the most common messages, resulting in better performance.

\subsection{Latency}

While latency is not a major concern for Magic Land%
\sidenote{
  The front end of Magic Land (written in Adobe Flash) takes several seconds
  to load. A difference for few milliseconds does not make an overall impact
  on the user experience.
}
, improving it is necessary since it blocks the client accessing it. Latency
can be improved by reducing the number of processes the proposal/message has
to travel through. However, this is quite tricky to optimize without breaking
the algorithm itself.

\section{Scalability}

In this context, scalability refers to increase the throughput of the system
by increasing the number of nodes in the cluster.

\subsection{Reads}

Reads are performed locally from the replicas itself and we do not use the
consensus protocol for it. This allows us to read directly from any of the
replicas.

It should be possible to create additional nodes that just subscribes to all
the decisions without taking part in the consensus protocol. We can then use
these nodes as read-only sources. This setup should allow us to scale the reads
to a fairly large extent.

\subsection{Writes}

Increasing the number of nodes in the cluster decreases its throughput since
more nodes are now involved in approving a proposal. Increasing number of writes
is hence a much harder task.

Couple of possible areas for improvements are:

\begin{itemize}
  \item Change the algorithm itself to reduce the total number of messages
    exchanged.
  \item Use different ways to broadcast messages. Such as ip-multicast as
    described in \citet{MarandiPSP10}
  \item Batch multiple request together to decrease individual round trip times.
\end{itemize}

\section{Back end Support}

Warlock's current back ends are key value stores. Since Warlock is transparent
to the commands, we can potentially use any data store as a back end. Adding
more types of back ends to Warlock can bring it to a wider audience.

\section{Recovery Techniques}

Currently if a Warlock node fails, we wipe all its data and restore it from
another node in the cluster using replication. While this method has its
advantages, there are cases where this might be a problem.

\begin{itemize}
  \item Certain back ends may be full blown disk recovery system. Allowing for
    conflict resolution between nodes can decrease the recovery time for a
    node drastically, especially for clusters with lot of data.
  \item Certain back ends may be provide the data backup and loading mechanism
    required by Warlock.
\end{itemize}

\section{Smart Client}

The current Warlock client is quite simple. It just connects to the server and
issues commands to it. Improving the client can allow us to decrease some
node on Warlock and improve availability. Some suggestions for such a improved
client are:

\begin{itemize}
  \item The client can store the list of Warlock nodes when it initiates the
    connection. This will allow the client to connect to other Warlock nodes
    (say in a round robin fashion) in case the one it is connected to goes down.
  \item Client can have certain timeout are provided by a Warlock node. Till
    allows us to control client timeouts from the server side.
  \item The Warlock nodes current route all the proposals they receive to the
    Master node. An alternative would be to have the client broadcast the
    request to all the nodes, resulting in reduction of traffic in the cluster.
  \item The client could queue proposals when the Warlock service is down for
    reasons such as leader election in progress or temporarily failed quorums.
  \item A technique used in \citet{Burrows06}, the client could cache the data
    and the master could actively invalidate cached data on updates. However,
    this could require a change in the current Warlock client server
    architecture, possibly making the system more complex.
\end{itemize}

\section{Expire}

Key expiry has many benefits, as discussed in \sectionref{a.n.d.expiry}. 
However, its implementation has severe trade-offs with performance. Hence, the
needs for performant key expiry exits.

\section{Others}

\subsection{Code improvement}

Several parts of the code that are open to improvement are tagged with 
``TODO''s. These are possible performance enhancement points, feature and
stability improvement points. 

For example, all the ETS tables used currently
are directly running under the processes that created them. If these processes
crash, the linked ETS table would also crash. Using a table manager would
help survive these crashes.

\subsection{More OTP}

Certain portions of the code skip some OTP practices to trade for performance.
For example, the commander is spawned directly by the leader instead of using
a supervisor. This makes it harder when it comes to hot code loading. In this
specific issue, it not much of a problem since commanders are short lived
process. A high level tuning from the OTP perspective can help improve code
stability and maintainability.

\subsection{Timeouts}

Warlock relies heavily in timers which are currently static. Making these
timeouts as a function of, say system load, can make the system more tolerant
of high loads.

\subsection{Reconfiguration}

The selection of nodes when transferring the master or when allocation the
seed node for replication is currently very simple. By using things such
as a priority list to keep track of the most stable nodes will allow us to
ensure more stable and operation.

\subsection{Shared Data}

Certain data within the Warlock consensus application is repeated between
processes. We currently do not share such data, making it simpler for us
to reason the system and allowing for localizing the failures. It might
be possible to improve performance trading for more complex code.

\subsection{Backups}

Warlock is designed to be an in-memory service with each node having a copy
of the entire dataset. This allows the database module to directly access
backup functions of the back end. The implementation of this feature should be
quite simple.

