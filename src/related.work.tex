\chapter{Related Work}
\label{chapter:related.work}

This chapter provides an overview of concepts related to this thesis. We
discuss relevant algorithms in brief and talk about similar projects. In the 
first section we discuss the Paxos family of algorithms and projects in 
this domain in the later sections.

\section{Paxos}
\label{section:paxos}

Paxos is regarded as the simplest and most obvious of distributed algorithms 
\citep{Lamport01}. It is a consensus protocol used for replication of state
machines in an asynchronous environment \citep{Lamport98}. We use Paxos in this
thesis primiarily since it has been shown that it has the minimal possible cost 
of any consensus algorithm in the presence of faults \citep{KeidarR03}.

A consensus algorithm tries to get a group of processes to agree on a value 
while satisfying its safety requirements%
\sidenote{
  Safety requirements of a consensus algorithm \citep{LamportM04}:
  \begin{inparaenum}[(i)]
    \item \emph{Nontriviality}: A value has to be proposed to be choosen.
    \item \emph{Consistency}: Different learners cannot learn different values.
    \item \emph{Conservatism}: Only choosen values can be learned and it can be
      learned atmost once.
  \end{inparaenum}
}. 
These processes in the Paxos algorithm can be classified based on their roles
without affecting its correctness:

\begin{itemize}
  \iterm{Proposer}: A process that can propose values to the group. 
  \iterm{Acceptor}: Acceptors form the ``memory'' of the algorithm to 
    allow choosing a single value.
  \iterm{Learner}: The choosen values are ``learned'' by the other 
    processes.
\end{itemize}

\begin{figure}
  \captionstyle{\raggedright}

      % Generated with LaTeXDraw 2.0.7
      % Tue Aug 14 11:05:02 CEST 2012
      % \usepackage[usenames,dvipsnames]{pstricks}
      % \usepackage{epsfig}
      % \usepackage{pst-grad} % For gradients
      % \usepackage{pst-plot} % For axes
      \scalebox{1.2} % Change this value to rescale the drawing.
      {
        \begin{pspicture}(0,-3.0226283)(9.481563,3.0626285)
          \psline[linewidth=0.04cm](0.5740625,2.5788784)(0.5940625,-3.0026283)
          \psline[linewidth=0.04cm](3.7740624,2.5788784)(3.7940626,-3.0026283)
          \psline[linewidth=0.04cm](4.5740623,2.5788784)(4.5940623,-3.0026283)
          \psline[linewidth=0.04cm](5.3740625,2.5788784)(5.3940625,-3.0026283)
          \psline[linewidth=0.04cm](8.174063,2.5788784)(8.194062,-3.0026283)
          \psline[linewidth=0.04cm](9.174063,2.5788784)(9.194062,-3.0026283)
          \usefont{T1}{ptm}{m}{n}
          \rput(0.6421875,2.8838785){Proposer}
          \usefont{T1}{ptm}{m}{n}
          \rput(4.5064063,2.8838785){Acceptors}
          \usefont{T1}{ptm}{m}{n}
          \rput(8.8142185,2.8838785){Learners}
          \psline[linewidth=0.024cm,arrowsize=0.113cm 2.0,arrowlength=1.4,arrowinset=0.4]{->}(0.5940625,2.1788783)(3.7940626,1.3788785)
          \psline[linewidth=0.024cm,arrowsize=0.113cm 2.0,arrowlength=1.4,arrowinset=0.4]{->}(0.5940625,2.1788783)(4.5940623,1.3788785)
          \psline[linewidth=0.024cm,arrowsize=0.113cm 2.0,arrowlength=1.4,arrowinset=0.4]{->}(0.5940625,2.1788783)(5.3940625,1.3788785)
          \psline[linewidth=0.024cm,arrowsize=0.113cm 2.0,arrowlength=1.4,arrowinset=0.4]{->}(3.7940626,0.97887844)(0.5940625,0.5788784)
          \psline[linewidth=0.024cm,arrowsize=0.113cm 2.0,arrowlength=1.4,arrowinset=0.4]{->}(4.5940623,0.97887844)(0.5940625,0.3788784)
          \psline[linewidth=0.024cm,arrowsize=0.113cm 2.0,arrowlength=1.4,arrowinset=0.4]{->}(5.3940625,0.97887844)(0.5940625,0.17887843)
          \psline[linewidth=0.024cm,arrowsize=0.113cm 2.0,arrowlength=1.4,arrowinset=0.4]{->}(0.5940625,-0.021121575)(3.7940626,-0.8211216)
          \psline[linewidth=0.024cm,arrowsize=0.113cm 2.0,arrowlength=1.4,arrowinset=0.4]{->}(0.5940625,-0.021121575)(4.5940623,-0.8211216)
          \psline[linewidth=0.024cm,arrowsize=0.113cm 2.0,arrowlength=1.4,arrowinset=0.4]{->}(0.5940625,-0.021121575)(5.3940625,-0.8211216)
          \psline[linewidth=0.024cm,arrowsize=0.113cm 2.0,arrowlength=1.4,arrowinset=0.4]{->}(3.7940626,-1.2211215)(8.194062,-1.8211216)
          \psline[linewidth=0.024cm,arrowsize=0.113cm 2.0,arrowlength=1.4,arrowinset=0.4]{->}(3.7940626,-1.2211215)(9.194062,-1.8211216)
          \psline[linewidth=0.024cm,arrowsize=0.113cm 2.0,arrowlength=1.4,arrowinset=0.4]{->}(4.5940623,-1.6211215)(8.194062,-2.2211215)
          \psline[linewidth=0.024cm,arrowsize=0.113cm 2.0,arrowlength=1.4,arrowinset=0.4]{->}(4.5940623,-1.6211215)(9.194062,-2.2211215)
          \psline[linewidth=0.024cm,arrowsize=0.113cm 2.0,arrowlength=1.4,arrowinset=0.4]{->}(5.3940625,-2.0211215)(8.194062,-2.6211216)
          \psline[linewidth=0.024cm,arrowsize=0.113cm 2.0,arrowlength=1.4,arrowinset=0.4]{->}(5.3940625,-2.0211215)(9.194062,-2.6211216)
          \usefont{T1}{ptm}{m}{n}
          \rput(2.2564063,2.0838785){\small P1a}
          \usefont{T1}{ptm}{m}{n}
          \rput(2.2603126,1.0838784){\small P1b}
          \usefont{T1}{ptm}{m}{n}
          \rput(2.2564063,-0.116121575){\small P2a}
          \usefont{T1}{ptm}{m}{n}
          \rput(6.8603125,-1.3161216){\small P2b}
        \end{pspicture} 
      }
      
      \caption[Basic Paxos]{%
        Basic Paxos Algorithm: Processes with roles - Proposer, Acceptor, 
        Learner - send messages to each other illustrating the flow of the 
        algorithm in a failure free instance.}
      \label{figure:basic_paxos}
  \normalcaption
\end{figure}

The algorithm proceeds in two phases with each phase having two sub phases.

\begin{itemize}
  \iterm{Phase 1 a}: Proposer select a number \emph{n} and sends it as a
  \emph{prepare} (P1A) message to all the acceptors.
  \iterm{Phase 1 b}: Acceptors compares the \emph{prepare} \emph{n} it receives
  and if it is greater than all previous numbers received as a part of 
  \emph{prepare}, it replies (P1B) with a promise not to accept any number lower
  than \emph{n}.
  \iterm{Phase 2 a}: If the proposer receives a response to its \emph{prepare}
  messages from a quorum%
  \sidenote{
    \emph{Quorum}: Majority agreement of processes. Quorum is used to ensure
    liviness in the system.
  }
  , it sends a reply (P2A) back to each of the acceptor with an \emph{accept} 
  message. The message also consists of the value \emph{v} which is the highest 
  numbered proposal among all the responses from the acceptors. In case it is 
  empty, the proposer is free to choose the value.
  \iterm{Phase 2 b}: If the acceptor has not received any \emph{prepare} request
  with a larger number when it receives an \emph{accept} request from the 
  proposer, it sends a message (P2B) to the learner with the accepted value 
  \emph{v}.
  \iterm{Learner}: If the learner receives a message from quorum of acceptor, it
  concludes that the value \emph{v} was choosen.
\end{itemize}

The algorithm makes progress when a proposed value is eventually learned by all
the learners. However, a scenario where no progress is made is possible when we 
have multiple proposers. Consider two proposers issuing \emph{prepare} requests
with alternatively increasing \emph{n}. They would keep pre-empting each other
in a loop leading do no agreement being reached among the group. A solution is
to create the role of \emph{distinguished proposer} or \emph{leader} where it
becomes the only process that can issue new requests. \citet{FisLynPat85}
implies the ``election'' of this \emph{leader} should use either randomness
or timeouts.

Although the pseudo-code of the algorithm is relatively small, implementing it
to create a stable, production ready system is non-trivial \citep{ChandraGR07}.
Different flavours of Paxos allows us to choose specific based on the project's
requirements.

The family of Paxos algorithms differ from each other based on the topology
of the process group, number of phases involved for one instance%
\sidenote{
  An instance of Paxos is a single run of the algorithm starting from the value
  being proposed to the learning of the value by the learners.
}
, amount of message delays and so on. We explore some of these Paxos variants.

\subsection{Basic Paxos}

Basic Paxos is the simplest version of Paxos and is the same as described 
previously in \sectionref{paxos}. The algorithm proceeds over several rounds
with the best case taking two rounds.

\subsection{Multi-Paxos}

The Paxos algorithm needs to be run multiple times for agreeing on a sequence of
values. \emph{Phase 1 a} and \emph{Phase 1 b} of the algorithm become 
an unnecessary overhead if the \emph{distinguished proposer} remains the same 
throught.

Multi-Paxos uses this as its basis to reduce the message flow. The first round 
of Multi-Paxos is the same as Basic Paxos. For subsequent values, the same
proposer starts directly with \emph{Phase 2 a} halving the message complexity.
Another proposer may take over at any point of time by starting with 
\emph{Phase 1 a} overriding the current proposer. This is not a problem since 
the original proposer can start again with \emph{Phase 1 a}.

\citet{Robbert2011} provides the imperative pseudo-code for Multi-Paxos and
the details required for making it practical. This thesis uses this paper as 
its basis for Paxos implementation.

\subsection{Fast Paxos}

Fast Paxos \citep{MSRTR2005112} is a variant of Basic Paxos. It has two
message delays compared to four message delays of Basic Paxos and guarantees
the round in three message delays in case of a collision.

Clients propose the values directly to the \emph{acceptors} and the 
\emph{leader} gets involved only in case there is a collision. Versions of Fast
Paxos can be optimized to a further extent by pre-specifling the collision
resolution technique allowing the clients to fix collisions themselves.

However, according to \citet{Vieira08theperformance} and \citet{Junqueira2007}
Fast Paxos is not better than Basic Paxos in all scenarios. Basic Paxos was 
found to be faster in case of systems with small number of replicas owing to 
the stablility provided by its use of a single co-ordinator and the variation
of message latencies in practical networks. Fast Paxos also needs larger quorum 
sizes of active replicas for it to function.

\subsection{Cheap Paxos}

Basic Paxos requires a total of \emph{2N + 1} servers in a distributed system 
even though \emph{N + 1} servers are enough (minimum) to make progress. 
Using servers that are slower or cheaper for the additional \emph{N} servers
allows us to reduce the total cost of the system. Cheap Paxos is designed along 
these lines.

Cheap Paxos uses \emph{N} auxilary servers along with \emph{N + 1} main servers
which allows it to tolerate \emph{N} failures. The idea is that the auxiliary
server steps to to replace one of the main servers when it goes down 
temporarily. The main server takes back control once restored. The auxiliary
servers thus act as a backup to the main servers without actively taking part
in the protocol, but merely acting as observers.

The downside of using Cheap Paxos is that it affects the liveliness of the 
system when multiple main servers fail at the same time since it takes time
for the auxillary servers to be configured into the system.

\subsection{Ring Paxos}

Ring Paxos \citep{MarandiPSP10} is based on the observations that messaging
using ip-multicast is more scalable and provides better throughput compared
to unicast for a distributed system with a well-defined network. It also
has the property that it provides fixed throughtput with variation in number
of receivers. It claims the throughput of ip-multicast and low latency of
unicast with the downside being that it provides weak synchrony.

\subsection{Stoppable Paxos}

Basic Paxos algorithm is run under the assumption that all the participating
processes are fixed and form a static system. However, the system should
support reconfiguation to be able to run for long periods of time in a
practical implementation. Reconfiguration includes adding new servers, 
removing/replacing old/faulty servers, scaling down the number of servers
when lower throughput is acceptable and so on.

Stoppable Paxos (\citet{LamportSP08}, \citet{LamportMZ10}) is one such algorithm
that allows us to reconfigure a Paxos based system. The algorithm defines a
special set of ``stopping'' commands. A stopping command is issued as the
\emph{i}th command after which no new command \emph{i+1} can be issued. The 
system proceeds normally after it executes the \emph{i}th command.

This thesis uses a variation of Stoppable Paxos for reconfiguration.

\subsection{Other}

Many other flavours of Paxos exists with some of them focussed on the 
implementation aspects of the algorithm. Lets take a look at some of them.

\begin{itemize}
    \iterm{Paxos for system builders}: \citet{Kirsch08paxosfor} provides the
    complete specification for implementing a system based on Paxos. It also
    details the performance, safety and liviness properties of a protoype built.
    \iterm{Paxos Made Live - An Engineering Perspective}: \cite{ChandraGR07}
    details the learning in engineering the Paxos algorithm for use in
    Google Chubby Locks \citep{Burrows06}.
\end{itemize}

\subsection{Implementations}

There are several implementations of Paxos and its variations. We describe full
featured implementations used in production in later sections. We discuss few
Erlang[[]] based implementations below since the thesis project is implemented
in Erlang.

\begin{itemize}
    \iterm{gen\_paxos}: \citep{Uenishi2012} implements Paxos with
    individual processes modelled as finite state machines. Each process can
    be performing a different role based on what state it is on. This 
    essentially makes all processes equal and ready to take on different roles
    as required during runtime.
    \iterm{LibPaxos}: \citet{Lugano2012} is a collection of open source 
    implementations of Paxos created specifically for performance measurements
    in \citet{MarandiPSP10}. It also includes a similator written in Erlang to
    observe the network behaviour.
    \iterm{gen\_leader}: \citet{Ulf2012} implements a leader election algorithm
    not based on Paxos. It is one of the notable implemetations for leader
    election in Erlang.
\end{itemize}

\section{Google chubby locks}
\label{section:chubby.locks}

Google created Chubby lock service \citep{Burrows06} that works like a 
distributed file system with advisory locks for loosely-coupled distributed 
systems. The goal of the project is to allow the clients to use the service
to synchronize their activities. For example, Google File System \citep{gfs} 
and BigTable \citep{ChangDGHWBCFG06} use Chubby locks for co-ordination and
as a store for small metadata \citep{ChandraGR07}.

Chubby lock is made up of two compoments:

\begin{itemize}
    \iterm{Chubby cell}: Chubby cell is typically made up of five servers 
    (termed replicas) that elect a master using Paxos protocol. The master
    server serves all the reads requests and co-ordinates the writes requests.
    The rest of the servers are for fault tolerance and are ready to replace the
    master should it fail.
    \iterm{Chubby client}: Chubby client maintains and open connection with
    the Chubby cell and communicates with it via RPC%
    \sidenote{
      \emph{RPC}: Remote Procedure Call: An inter-process communication
      technique where one process can run programs on remote processes.
    }
    . The client maintains an in-memory write though cache that
    is kept consistent by the master using invalidations. The client is
    aware of the cell status using special requests%
    \sidenote[1]{
      \emph{KeepAlives}: Periodic requests used for indicating status.
    }
    .
\end{itemize}

A Chubby replica mainly consists of a fault tolerant log that is consistent
with other replicas in the cell by using Paxos protocol. The rest of the
replica is made of a fault tolerant database and an interface to handle
requests from Chubby clients. The specfic Paxos flavour used is Multi-Paxos
with slight modifications such as having a ``catch-up'' mechanism for
slower replicas.

The presence of Chubby locks project and its use in some of the largest 
server installations is a testimony for the need for distributed lock
managers. This thesis uses some of the ideas explored in Chubby locks.

\section{Google Megastore}

Megastore \citep{BakerBCFKLLLLY11} is an ACID%
\sidenote{
  ACID properties are used to provide guarantees for database transactions.
  \begin{inparaenum}[(i)]
    \iterm{Atomicity}: A transaction is either executed completely or not 
    executed at all.
    \iterm{Consistency}: The state of the database remains consistent after
    the transaction has been completed.
    \iterm{Isolation}: Transactions executed in parallel results in the same
    state as running all the transactions serially.
    \iterm{Durable}: All changes made by a transaction to a database is
    permanent.
  \end{inparaenum}
}
compliant, scalable datastore that guarantees high availability and
consistency. It uses synchronous replication for high availibility and
consistenct views while targeting performance by parititioning the data.

Megastore users Paxos to replicate a write-ahead log, replicate commit
records for single phase ACID transactions and as a part of fast failover
mechanisms. It provides fast local reads using a service called the
\emph{co-ordinator} which keeps track of the data version/Paxos write
sequence over the group of replicas. It speeds up writes by pre-preparing
optimizations and other heuristics.

\section{Zookeeper}
\label{section:zookeeper}

Zookeeper \citep{Hunt:2010:ZWC:1855840.1855851, zookeeper} is a open source 
consensus service written in java that is used for synchronization in 
distributed applications and as a metadata store. It is inspired by Chubby lock 
\sectionref{chubby.locks}, but uses its own protocol Zookeeper Atomic Broadcast 
(ZAB) in place of Paxos.

\subsection{Zookeeper Atomic Broadcast}

Zookeeper Atomic Broadcast (ZAB) 
\citep{Reed:2008:STO:1529974.1529978, JunqueiraRS11} is a 
totally ordered atomic broadcast protocol created for usage in Zookeeper. ZAB
satisfies the constraints imposed by Zookeeper viz.,

\begin{itemize}
    \iterm{Reliable delivery}: Message delived to one server must eventually
    get delivered to all the servers.
    \iterm{Total order}: Every server should see the same ordering of the 
    delivered messages.
    \iterm{Causal order}: Messages should follow causal%
    \sidenote{
      If message \emph{a} is delivered before message \emph{b} on a server then
      all other servers in the group should receive message \emph{a} before 
      \emph{b}.
    }
    ordering.
\end{itemize}

ZAB is conceptually similar to Paxos, but uses certain optimizations and 
tradeoffs. The service using ZAB has two modes of operation:

\begin{itemize}
    \iterm{Broadcast mode}: Broadcast mode starts when a new leader is choosen
    using a quorum from the group. The leader's state is now same as rest of
    the servers and can hence start broadcasting messages.
    This mode is similar to two-phase commitn \citep{Gray78}, but with quorum.
    \iterm{Recovery mode}: A new leader has to choosen when the exiting leader 
    is no longer valid. The service is now in recovery mode until a new leader
    emerges.
\end{itemize}

Zookeeper uses the concept of ``observers'' to increase read throughput. They 
are a set of servers that monitor the zookeeper service, but do not take part
in the protocol directly thus acting as extended replicas. The write throughput
is however inversly proportional to the number of servers in the group mainly
due to the increased co-ordination required for concensus.

Zookeeper worked on a static set of servers with no option to reconfigure the
cluster at the time of writing. However, the feature is in the works 
\citep{zab2012}and an initial release is avaiable.

\section{Doozerd}

Doozerd \citep{doozerd} is a consensus service similar to Chubby locks
\sectionref{chubby.locks} and Zookeeper \sectionref{zookeeper} written in 
Go \citep{golang}. It uses Paxos protocol internally for maintaining
write consistency.

\section{Riak}

Riak \citep{riak} is a distributed eventually consistent datastore written in 
Erlang. It is based on the concepts from Amazon's Dynamo \citep{DeCandia07}.

\section{Amazon Dynamo}

Dynamo \citep{DeCandia07} is a eventually consistent key-value store developed 
inside Amazon designed to be incrementally scalable and highly available.

\section{Dynamo DB}

Amazon DynamoDB \citep{dynamoDB} is proprietary key-value datastore based on
\citet{DeCandia07} available as a service.

\section{Scalaris}

Scalaris \citep{scalaris} is distributed key-value database the supports
consistent writes and full ACID properties. It is based on Distribtued Hash
Table (DHT)%
\sidenote{
  \emph{Distributed Hash Table}: DHT is a distributed system that provides
  hash table operations.
} 
and is implemented in Erlang.



