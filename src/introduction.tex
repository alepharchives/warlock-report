\chapter{Introduction}
\label{chapter:introduction}

Shared nothing architecture%
\sidenote[7]{
  \emph{Shared Nothing}: A type of architecture where the servers do not share
  any hardware resources between them.
} \citep{Stonebraker86g} is a popular distributed system design used for
building high traffic web applications. The servers within such a system setup
sometimes need to synchronize their activities or to access a shared
resource. A service used for such a purpose could act as a single point of
failure based on its design and the system architecture.

The online social game Magic Land uses a stateful application server
architecture to handle millions of users. This architecture uses Redis as
a global registry to keep track of user sessions. The registry acts as a
locking mechanism allowing for at most of one user session at a time. Failure
of this software, or the server it is deployed on can
lead to application downtime. This creates a need for
building a service that is itself
distributed, fault tolerant and, primarily, consistent.

\section{Goals}

The goal of this thesis is to build a customized locking service focused on
being fault tolerant without compromising consistency or performance.
Specifically we require,

\begin{itemize}
    \iterm{Strong consistency}: The state of the system remains valid across
    all concurrent requests.
    \iterm{High availability}: The system should be operable at all times.
    \iterm{Performance}: The system should be able to handle a large number of
    requests.
    \iterm{Fault tolerant}: Server failures should be isolated and the system
    should continue to function despite these failures.
\end{itemize}

This locking mechanism can be realized by using state machines replicated
across multiple servers which are kept consistent using distributed consensus
algorithms. We investigate such algorithms and analyze similar
projects. We then create a system design based on these observations and
a working implementation.

\section{Overview}

Consensus in distributed systems is a large topic in itself. Consensus
algorithms are notoriously hard to implement even though the pseudo code
describing them can be relatively simple. Among the set of algorithms
available for arriving at consensus in an asynchronous distributed
system, we choose Paxos because of its literature coverage and usage
in the industry.

The Erlang programming language is built with the primitives necessary for
creating a distributed system, such as stable messaging, crash tolerance and
well defined behaviours. The Actor model of Erlang processes maps very well to
the algorithms pseudo code. Using Erlang allows us to separate the details of
the algorithm from the implementation specifics. This enables us to ensure
accuracy and makes debugging simpler.

The idea behind the system design is that we use Multi-Paxos%
\sidenote{
  \emph{Multi-Paxos}: A variant of the Paxos algorithm designed to reach
  consensus wtih fewer number of messages and steps.
} for consensus, reconfigurable state machine algorithms to make the
cluster dynamic and implement this in Erlang.

The Erlang implementation of this thesis is named \term{Warlock}.

\section{Organization}
The thesis is divided into three main sections over multiple chapters.

The first section provides the context and background information related to the
project. It details the problem background, related projects, the research
area and the set of requirements the project is based on.

The second section describes the analysis, design and implementation of
the project, and the experiments performed on it.

The final section discusses the project results and provides an outline for
future work.

\section{Scope}

The scope of the project is to implement a reasonably stable application that
satisfies the above goals and can be used in production. The application needs
to have good test coverage and documented code. The thesis scope does not cover
creation of any new algorithms or techniques, but rather builds on the basis of
well known ideas in the field.

