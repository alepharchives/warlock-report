\chapter{Introduction}
\label{chapter:introduction}

\section{Problem}

Distributed software systems sometimes use services that allow them to 
synchronize their activities or to access a shared resource. Such a
service could itself act as a single point of failure depending on the
application architecture.

Online social game Magic Land uses a stateful application server architecture to
handle millions of users. This architecture uses Redis as a global registry to
keep track of user sessions. Failure of the software Redis, or the machine it
runs on can lead to application downtime.

\section{Goals}

The main goals for the design and implementation of a locking service to address
the problem are

\begin{itemize}
  \item Strong consistency
  \item High availability
  \item Performance
  \item Fault tolerant
\end{itemize}

We will investigate the algorithms for distributed consensus and analyze similar
projects. We will then create a system design based on these observations and
an working implementation.

\section{Scope}

The scope of the project is to implement a reasonably stable application that
satisfies the above goals and can be used in production. The application needs
to have good test coverage and documented code. The thesis scope does not cover
creation of any new algorithms or techniques, but rather builds on the basis of
well known ideas in the field.


\section{Outline}
The thesis is divided into three main sections over multiple chapters.

Firstly, we provide the context of the project, separated as:

\begin{itemize}
    \iterm{Background}: The background chapter provides us with an idea of the
    problem domain and the set of expected challenges when designing a
    distributed system.
    \iterm{Related Work}: We cover the related algorithms and projects in the
    problem domain in this chapter. We also detail why we cannot existing
    solutions using comparisons.
    \iterm{Requirements}: We specify the project requirements and how it
    related to the project in this chapter.
\end{itemize}

Secondly, we detail the work done on the thesis as:

\begin{itemize}
    \iterm{Concepts}: We cover the important domain concepts required to
    understand the solution in this chapter.
    \iterm{Analysis and Design}: We observe all the analysis done for the
    project and detail the design resulting from this analysis.
    \iterm{Implementation}: This chapter covers specific details in terms of
    how the project was implemented and how the problem maps to the Erlang
    language.
    \iterm{Experiments and Performance}: We take a look at all the performance
    measurements and different experiments conducted to ascertain the
    application's capabilities in this chapter.
\end{itemize}

Lastly, we wrap up with a summary of the thesis:

\begin{itemize}
    \iterm{Conclusion}: We provide a short description of what the project has
    achieved and an top down view of all the observations made.
    \iterm{Future Work}: A list of improvements, in terms of features and
    performance that can be done to the application is listed in this chapter.
\end{itemize}


The first section provides the context and background information related to the
project. It details the problem background, related projects, the research
area and the set of requirements the project is based on.

The second section describes the analysis, design, implementation of the project
and the experiments performed on it.

The final section discusses the project results and provides an outline for
the future work.

