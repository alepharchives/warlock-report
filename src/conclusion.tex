\chapter{Conclusion}
\label{chapter:conclusion}

The work done on this thesis can be classified into four areas.

\begin{enumerate}
  \item We have surveyed a number of papers and projects related to distributed
    lock management and compared them in the context of our requirements.
  \item We designed a system based on the literature survey and prototyping.
  \item We implemented a working system in Erlang.
  \item We tested the system to make sure it satisfies the specified
    requirements and can sustain it.
\end{enumerate}

We were able to select a specific flavour of Paxos and optimize it, use other
state machine algorithms, pick ideas from other projects in the field and build 
a scaffolding around it to create the lock manager. 

Separating the implementation from the algorithm details turned out
to be very essential in ensuring the correctness of the implementation. Erlang
provided us with robust and well tested distributed programming primitives. 
These led to creation of a modular architecture which enables us to extend
it to be used in wider application domains.

The graphs indicating performance over time are all simple and remain flat
across signifying steady operation. The results of this study indicate that
we were able to achieve our goal of creating a strongly consistent
distributed lock manager.

